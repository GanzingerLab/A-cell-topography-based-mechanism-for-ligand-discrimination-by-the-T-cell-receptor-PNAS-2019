\documentclass[11pt]{article}

\usepackage{amsmath, amssymb, array, color, dcolumn, times, graphicx, rotating,multicol,multirow}
\usepackage[letterpaper, margin=0.8in]{geometry}

%\usepackage[utf8x]{inputenc}
\usepackage{setspace}
\usepackage{pstricks}
%\usepackage{sidecap}
%\usepackage{url}
\usepackage{caption}
\captionsetup[figure]{labelfont=bf}

%\usepackage[super,numbers,round, sort&compress]{natbib} 

\usepackage[numbers, round, sort&compress]{natbib} 

\bibliographystyle{pnas}
\renewcommand{\bibnumfmt}[1]{#1.}


\setlength{\parskip}{0.125in}
\setlength{\parindent}{0pt}

\setlength{\columnsep}{12pt}
\setlength{\headheight}{13.6pt}

\usepackage{graphicx}

\newcommand{\kon}{k_\text{on}}
\newcommand{\kons}{k_\text{on}^*}
\newcommand{\koff}{k_\text{off}}
\newcommand{\kt}{k_\text{cat}}
\newcommand{\km}{K_\text{m}}
\newcommand{\kc}{k_\text{cat}}
\newcommand{\kcsurf}{k_\text{cat}^*\ \text{(tethered)}}
\newcommand{\kcsol}{k_\text{cat}^*\ \text{(solution)}}
\newcommand{\kd}{K_\text{D}}
\newcommand{\dc}{$^\circ$C}
\newcommand{\vm}{V_\text{max}}

\newcommand{\br}{\textbf{r}}

\newcommand{\jt}[1]{\textcolor{black}{#1}}
\newcommand{\al}[1]{\textcolor{black}{#1}}

\newcommand{\bsub}{\begin{subequations}}
\newcommand{\esub}{\end{subequations}$\!$}

\title{Model for receptor triggering in close contact zones}

\author{}
\date{}


\begin{document}

\maketitle


\begin{abstract}
\end{abstract}


%\textbf{Running title.} \\

%\textbf{Key words.} \\
\clearpage

\subsection*{Model description}

Close contact zones (CCZs) are formed at the interface between a T cell and another cell. They are generally small in size but grow over time. Importantly, they are only formed for finite periods of time (referred to as the contact duration) after which they disassemble. When a CCZ is formed, T cell receptors (TCRs) are able to diffuse in and out of these zones. However, if the TCR binds to ligand within the CCZ, although continuing to diffuse within the CCZ, the bound complex is unable to leave. When TCR remain within the CCZ for longer than 2 seconds, independent of their binding status, they are said to be triggered. The model aims to calculate how the triggering probability depends on CCZ size, growth, and duration when ligands are presented at different doses and affinities.

\subsection*{Model formulation}

The dynamics of TCR in a CCZ is governed by a non-linear moving-boundary coupled partial-differential-equations (PDEs),
\begin{eqnarray}
\partial T / \partial t &=& D_T \nabla^2 T - \kon^* T + \koff C, \qquad \al{0 < |\br| < R(t;t_\textrm{entry});}\\
\partial C / \partial t &=& D_C \nabla^2 C + \kon^* T - \koff C,  \qquad \al{0 <|\br| < R(t;t_\textrm{entry}),}
\end{eqnarray}
where $T(\mathbf{r},t; t_\textrm{entry})$ and $C(\mathbf{r},t; t_\textrm{entry})$ represent the free and ligand-complexed TCR that diffuse with coefficient $D_T$ and $D_C$, respectively, and undergo reversible binding with first-order rates ($\kon^*$, $\koff$). Note that $\kon^* = \kon [M]$ where $\kon$ is the bimolecular on-rate (in units of $\mu$m$^{2}$/s) and $[M]$ is the ligand concentration (in units of $\mu$m$^{-2}$). The boundary conditions for the disc domain of radius $R$ are adsorbing for $T$ and \al{no flux} for $C$,
\bsub\label{BCs}
\begin{eqnarray}
\label{BC_a} T(R) &=& 0,  \\
\label{BC_b} \al{D_C \nabla C \cdot \hat{\textbf{n}} + R^\prime(t)C}  &\jt{=}& \jt{0}
\end{eqnarray}
\esub
%
Importantly, the domain area \al{grows linearly} in time and therefore,
\begin{eqnarray}
R(t;t_{\textrm{entry}}) = \sqrt{\jt{R_0^2 + } g (t + t_\textrm{entry}) / \pi} 
\end{eqnarray}
where $g$ is the growth rate (in units of $\mu$m$^2$/s) and $t$ is time. The initial conditions at $t = t_\textrm{entry}$ are as follows,
\bsub\label{ICs}
\begin{eqnarray}
\label{IC_a} T(r) &=& \delta(\br - \br_0); \\
\label{IC_b} C(r) &=& 0,
\end{eqnarray}
\esub
\al{where $\br_0 =  (\jt{R_0}-\epsilon,\al{\theta})$. The additional term $R^\prime(t)C $ in \eqref{BC_b}, which reflects the rate of growth in the region, is a necessary addition to the usual Neumann condition in order to prevent mass of $C$ leaving the domain. To see this, consider the change in total mass $M(t) = \int_{\Omega(t)}(T+C)d\br$,}
\begin{align}
\nonumber M'(t) = \frac{d}{dt} \int_{0}^{2\pi} \int_0^{R(t)} (T + C) r dr d\theta &= R R' \int_{0}^{2\pi} [T + C]_{r=R}\, d\theta + \int_{0}^{2\pi} \int_0^{R(t)} \frac{d}{dt} (T + C) r dr d\theta,\\
\nonumber {} & = \int_{0}^{2\pi}  [R R' (T + C) + R ( D_T T_r+ D_C C_r)]_{r=R}\, d\theta\\
\nonumber {} & = R\int_{0}^{2\pi}  [  D_C C_r + R' C  ]_{r=R} \, d\theta +    D_T R \int_{0}^{2\pi} [T_r]_{r=R} \, d\theta
\end{align}
\al{The flux of T-cell receptors in complex ($C$) through the boundary should be zero which gives rise to the boundary condition \eqref{BC_b}}

\subsection*{Model output}

The output of the model is the probability ($P_s$) that a single receptor has remained within the domain for more than 2 seconds during the contact duration ($t_f$),
\begin{eqnarray}
%P_s(t_\textrm{entry}) = \jt{-}\int_2^{t_{f}} dt \jt{\frac{\partial}{\partial t}}\int_{\Omega(t)} d\textbf{r}  \left( T(t,\textbf{r}) + C(t,\textbf{r}) \right) \jt{ = \int_{\Omega(2)} \! T(2, \mathbf{r}) + C(2, \mathbf{r}) \, d\mathbf{r} - \int_{\Omega(t_f)} \! T(t_f, \mathbf{r}) + C(t_f, \mathbf{r}) \, d\mathbf{r}}
P_s(t_\textrm{entry}) = \int_{\Omega(2;t_\textrm{entry})} \! T(\mathbf{r}, 2; t_\textrm{entry}) + C(\mathbf{r}, 2; t_\textrm{entry}) \, d\mathbf{r}.
\end{eqnarray}

The time-dependent rate of TCR entry into the domain ($k_t (t)$) is expected to be proportional to the size of the domain, which increases over time. Using previously derived results (see Equations 11 in Weaver (1983) Diffusion-mediated localisation on membrane surfaces, Biophysical Journal), we find,
\begin{eqnarray}
k_t(t) = \frac{4\pi D T_m}{\log(A / (\pi R(t)^2) - 1) }
\end{eqnarray}
where $A = 415$ $\mu$m$^2$ is the cell surface area, $T_m = 100$ $\mu$m$^{-2}$ is the TCR concentration far away from the CCZ, and $D_T = 0.05$ $\mu$m$^2$/s is the TCR diffusion coefficient. With these numbers, we find that the rate of TCR entry into the domain ($k_t$) increases from $\approx$4 s$^{-1}$ to $\approx$18 s$^{-1}$ as the domain radius increases from 0.01 $\mu$m to 2 $\mu$m.

Given that multiple receptors can enter the close contact during the contact duration ($t_f$), we need to calculate the probability that at least one TCR has remained within the domain for more than 2 ($P_m$). The number of TCRs that have entered the domain in time interval $\lbrack t_i, t_i + \Delta t\rbrack$ can be estimated as $k_t(t_i)\Delta t$ so that $P_m$ is estimated as follows,
\begin{eqnarray} \label{Pm}
P_m = 1 - \prod_{i=1}^{N} \left[ 1-P_s(t_i)\right]^{k_t(t_i)\Delta t} \,.
\end{eqnarray}
In the case where $P_s$ and $k_t$ are constants
\begin{eqnarray} \label{Pm}
P_m = 1 - \prod_{i=1}^{N} \left[ 1-P_s\right]^{k_t\Delta t} \, = 1 -  \left[ 1-P_s\right]^{k_tN\Delta t} = 1 - \left[ 1-P_s\right]^{k_t(t_f-2)}
\end{eqnarray}

\subsection*{Implemetation (MATLAB)}

The function \textbf{MFPTLoop} (calling out MFPTGrowingDisc) calculates the probability ($P_s$) that a single receptor has remained within the domain for more than 2 seconds for different scenarios (corresponding to the Figures in Fernandes et al., 2019; MFPT stands for mean free passage time). \textbf{GetPm} uses the output from this function ($P_s$ values) to calculate $P_m$ (``Triggering probability'') for the same scenarios.
%\begin{figure}[!htb]\centering
%\includegraphics[width=10cm]{FIG01.eps}
%\caption{}
%\label{fig:solution}
%\end{figure}


%\bibliography{/Users/omer/Dropbox/Work/hanako/reference_library/mendeley_importhome/library}





\end{document}










